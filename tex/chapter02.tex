%# -*- coding: utf-8-unix -*-
%%==================================================
%% chapter02.tex for SJTU Master Thesis
%% Encoding: UTF-8
%%==================================================

\chapter{ \LaTeX 排版例子}
\label{chap:example}

\section{数学排版}
\label{sec:matheq}

\subsection{公式排版}
\label{sec:eqformat}

这里有举一个长公式排版的例子,来自\href{http://www.tex.ac.uk/tex-archive/info/math/voss/mathmode/Mathmode.pdf}{《Math mode》}:

\begin {multline}
  \frac {1}{2}\Delta (f_{ij}f^{ij})=
  2\left (\sum _{i<j}\chi _{ij}(\sigma _{i}-
    \sigma _{j}) ^{2}+ f^{ij}\nabla _{j}\nabla _{i}(\Delta f)+\right .\\
  \left .+\nabla _{k}f_{ij}\nabla ^{k}f^{ij}+
    f^{ij}f^{k}\left [2\nabla _{i}R_{jk}-
      \nabla _{k}R_{ij}\right ]\vphantom {\sum _{i<j}}\right )
\end{multline}

\subsubsection{一个四级标题}
\label{sec:depth4}

这是全文唯一的一个四级标题。在这部分中将演示了mathtools宏包中可伸长符号(箭头、等号的例子)的例子。

\begin{displaymath}
    A \xleftarrow[n=0]{} B \xrightarrow[LongLongLongLong]{n>0} C 
\end{displaymath}

\begin{eqnarray}
  f(x) & \xleftrightarrow[]{A=B}  & B \\
  & \xleftharpoondown[below]{above} & B \nonumber \\
  & \xLeftrightarrow[below]{above} & B
\end{eqnarray}

又如:

\begin{align}
  \label{eq:none}
  & I(X_3;X_4)-I(X_3;X_4\mid{}X_1)-I(X_3;X_4\mid{}X_2) \nonumber \\
  = & [I(X_3;X_4)-I(X_3;X_4\mid{}X_1)]-I(X_3;X_4\mid{}\tilde{X}_2) \\
  = & I(X_1;X_3;X_4)-I(X_3;X_4\mid{}\tilde{X}_2)
\end{align}

\subsection{定理环境}

模板中定义了丰富的定理环境
algo(算法),thm(定理),lem(引理),prop(命题),cor(推论),defn(定义),conj(猜想),exmp(例),rem(注),case(情形),
bthm(断言定理),blem(断言引理),bprop(断言命题),bcor(断言推论)。
amsmath还提供了一个proof(证明)的环境。
这里举一个“定理”和“证明”的例子。
\begin{thm}[留数定理]
\label{thm:res}
  假设$U$是复平面上的一个单连通开子集,$a_1,\ldots,a_n$是复平面上有限个点,$f$是定义在$U\backslash \{a_1,\ldots,a_n\}$上的全纯函数,
  如果$\gamma$是一条把$a_1,\ldots,a_n$包围起来的可求长曲线,但不经过任何一个$a_k$,并且其起点与终点重合,那么:

  \begin{equation}
    \label{eq:res}
    \ointop_{\gamma}f(z)\,\mathrm{d}z = 2\uppi\mathbf{i}\sum^n_{k=1}\mathrm{I}(\gamma,a_k)\mathrm{Res}(f,a_k)
  \end{equation}

  如果$\gamma$是若尔当曲线,那么$\mathrm{I}(\gamma, a_k)=1$,因此:

  \begin{equation}
    \label{eq:resthm}
    \ointop_{\gamma}f(z)\,\mathrm{d}z = 2\uppi\mathbf{i}\sum^n_{k=1}\mathrm{Res}(f,a_k)
  \end{equation}

      % \oint_\gamma f(z)\, dz = 2\pi i \sum_{k=1}^n \mathrm{Res}(f, a_k ). 

  在这里,$\mathrm{Res}(f, a_k)$表示$f$在点$a_k$的留数,$\mathrm{I}(\gamma,a_k)$表示$\gamma$关于点$a_k$的卷绕数。
  卷绕数是一个整数,它描述了曲线$\gamma$绕过点$a_k$的次数。如果$\gamma$依逆时针方向绕着$a_k$移动,卷绕数就是一个正数,
  如果$\gamma$根本不绕过$a_k$,卷绕数就是零。

  定理\ref{thm:res}的证明。
  
  \begin{proof}
    首先,由……

    其次,……

    所以……
  \end{proof}
\end{thm}

上面的公式例子中,有一些细节希望大家注意。微分号d应该使用“直立体”也就是用mathrm包围起来。
并且,微分号和被积函数之间应该有一段小间隔,可以插入\verb+\,+得到。
斜体的$d$通常只作为一般变量。
i,j作为虚数单位时,也应该使用“直立体”为了明显,还加上了粗体,例如\verb+\mathbf{i}+。斜体$i,j$通常用作表示“序号”。
其他字母在表示常量时,也推荐使用“直立体”譬如,圆周率$\uppi$(需要upgreek宏包),自然对数的底$\mathrm{e}$。
不过,我个人觉得斜体的$e$和$\pi$很潇洒,在不至于引起混淆的情况下,我也用这两个字母的斜体表示对应的常量。


\section{向文档中插入图像}
\label{sec:insertimage}

\subsection{支持的图片格式}
\label{sec:imageformat}

\XeTeX 可以很方便地插入PDF、EPS、PNG、JPG格式的图片。

插入PNG/JPG的例子如\ref{fig:SRR}所示。
这两个水平并列放置的图共享一个“图标题”(table caption),没有各自的小标题。

\begin{figure}[!htp]
  \centering
  \includegraphics[width=0.3\textwidth]{chap2/sjtulogo.png}
  \hspace{1cm}
  \includegraphics[width=0.3\textwidth]{chap2/sjtulogo.jpg}
  \bicaption[fig:SRR]{这里将出现在插图索引中}{中文题图}{Fig}{English caption}
\end{figure}

这里还有插入eps图像和pdf图像的例子,如图\ref{fig:epspdf:a}和图\ref{fig:epspdf:b}。这里将EPS和PDF图片作为子图插入,每个子图有自己的小标题。并列子图的功能是使用subfigure宏包提供的。

\begin{figure}
  \centering
  \subfigure[EPS Figure]{
    \label{fig:epspdf:a} %% label for first subfigure
    \includegraphics[width=0.3\textwidth]{chap2/sjtulogo.eps}}
  \hspace{1in}
  \subfigure[PDF Figure]{
    \label{fig:epspdf:b} %% label for second subfigure
    \includegraphics[width=0.3\textwidth]{chap2/sjtulogo.pdf}}
  \bicaption[fig:pdfeps]{插入eps图像和pdf图像}{插入eps和pdf的例子}{Fig}{An EPS and PDF demo}
\end{figure}

更多关于 \LaTeX 插图的例子可以参考\href{http://www.cs.duke.edu/junhu/Graphics3.pdf}{《\LaTeX 插图指南》}。

\subsection{长标题的换行}
\label{sec:longcaption}

图\ref{fig:longcaptionbad}和图\ref{fig:longcaptiongood}都有比较长图标题,通过对比发现,图\ref{fig:longcaptiongood}的换行效果更好一些。
其中使用了minipage环境来限制整个浮动体的宽度。

\begin{figure}[!htp]
 \centering
 \includegraphics[width=4cm]{chap2/sjtulogo.pdf}
 \bicaption[fig:longcaptionbad]{这里将出现在插图索引}{上海交通大学是我国历史最悠久的高等学府之一,是教育部直属、教育部与上海市共建的全国重点大学.}{Fig}{Where there is a will, there is a way.}
\end{figure}

\begin{figure}[!hbp]
  \centering
  \begin{minipage}[b]{0.6\textwidth}
    \captionstyle{\centering}
    \centering
    \includegraphics[width=4cm]{chap2/sjtulogo.pdf}
    \bicaption[fig:longcaptiongood]{这里将出现在插图索引}{上海交通大学是我国历史最悠久的高等学府之一,是教育部直属、教育部与上海市共建的全国重点大学.}{Fig}{Where there is a will, there is a way.}
  \end{minipage}     
\end{figure}
  
\clearpage

\section{表格}
\label{sec:tab}

这一节给出的是一些表格的例子,如表\ref{tab:firstone}所示。

\begin{table}[!hpb]
  \centering
  \bicaption[tab:firstone]{指向一个表格的表目录索引}{一个颇为标准的三线表格\footnotemark[1]}{Table}{A Table}
  \begin{tabular}{@{}llr@{}} \toprule
    \multicolumn{2}{c}{Item} \\ \cmidrule(r){1-2}
    Animal & Description & Price (\$)\\ \midrule
    Gnat & per gram & 13.65 \\
    & each & 0.01 \\
    Gnu & stuffed & 92.50 \\
    Emu & stuffed & 33.33 \\
    Armadillo & frozen & 8.99 \\ \bottomrule
  \end{tabular}
\end{table}
\footnotetext[1]{这个例子来自\href{http://www.ctan.org/tex-archive/macros/latex/contrib/booktabs/booktabs.pdf}{《Publication quality tables in LATEX》}(booktabs宏包的文档)。这也是一个在表格中使用脚注的例子,请留意与threeparttable实现的效果有何不同。}

下面一个是一个更复杂的表格,用threeparttable实现带有脚注的表格,如表\ref{tab:footnote}。

\begin{table}[!htpb]
  \bicaption[tab:footnote]{出现在表目录的标题}{一个带有脚注的表格的例子}{Table}{A Table with footnotes}
  \centering
  \begin{threeparttable}[b]
     \begin{tabular}{ccd{4}cccc}
      \toprule
      \multirow{2}{6mm}{total}&\multicolumn{2}{c}{20\tnote{1}} & \multicolumn{2}{c}{40} &  \multicolumn{2}{c}{60}\\
      \cmidrule(lr){2-3}\cmidrule(lr){4-5}\cmidrule(lr){6-7}
      &www & k & www & k & www & k \\
      \midrule
      &$\underset{(2.12)}{4.22}$ & 120.0140\tnote{2} & 333.15 & 0.0411 & 444.99 & 0.1387 \\
      &168.6123 & 10.86 & 255.37 & 0.0353 & 376.14 & 0.1058 \\
      &6.761    & 0.007 & 235.37 & 0.0267 & 348.66 & 0.1010 \\
      \bottomrule
    \end{tabular}
    \begin{tablenotes}
    \item [1] the first note.% or \item [a]
    \item [2] the second note.% or \item [b]
    \end{tablenotes}
  \end{threeparttable}
\end{table}

\section{参考文献管理}

 \LaTeX 具有将参考文献内容和表现形式分开管理的能力,涉及三个要素:参考文献数据库、参考文献引用格式、在正文中引用参考文献。
这样的流程需要多次编译:

\begin{enumerate}[noitemsep,topsep=0pt,parsep=0pt,partopsep=0pt]
	\item 用户将论文中需要引用的参考文献条目,录入纯文本数据库文件(bib文件)。
	\item 调用xelatex对论文模板做第一次编译,扫描文中引用的参考文献,生成参考文献入口文件(aux)文件。
	\item 调用biber,以参考文献格式和入口文件为输入,生成格式化以后的参考文献条目文件(bib)。
	\item 再次调用xelatex编译模板,将格式化以后的参考文献条目插入正文。
\end{enumerate}

参考文献数据库(thesis.bib)的条目,可以从Google Scholar搜索引擎\footnote{\url{https://scholar.google.com}}、CiteSeerX搜索引擎\footnote{\url{http://citeseerx.ist.psu.edu}}中查找,文献管理软件Papers\footnote{\url{http://papersapp.com}}、Mendeley\footnote{\url{http://www.mendeley.com}}、JabRef\footnote{\url{http://jabref.sourceforge.net}}也能够输出条目信息。

下面是在Google Scholar上搜索到的一条文献信息,格式是纯文本:

\begin{lstlisting}[caption={从Google Scholar找到的参考文献条目}, label=googlescholar, float, escapeinside="", numbers=none]
    @phdthesis{"白2008信用风险传染模型和信用衍生品的定价",
      title={{"信用风险传染模型和信用衍生品的定价"}},
      author={"白云芬"},
      year={2008},
      school={"上海交通大学"}
    } 
\end{lstlisting}

为了方便引用,建议将文献别名改用ASCII表示。
中文文献的处理有别于英文文献,对中文参考文献需要增加“language”域显式说明。

\begin{lstlisting}[caption={修改后的参考文献条目}, label=itemok, float, escapeinside="", numbers=none]
  @phdthesis{bai2008,
    title={{"信用风险传染模型和信用衍生品的定价"}},
    author={"白云芬"},
    year={2008},
    language={chinese},
    address={"上海"},
    school={"上海交通大学"}
  } 
\end{lstlisting}

按照教务处的要求,参考文献外观应符合国标GBT7714的要求\footnote{\url{http://www.cces.net.cn/guild/sites/tmxb/Files/19798_2.pdf}}。
在本模板中,我们使用caspervector\footnote{\url{http://www.ctan.org/pkg/biblatex-caspervector}}参考文献样式,这是一种非常接近GB风格,但在美学细节上更优秀的风格。

正文中引用参考文献时,用\verb+\supercite{key1,key2,key3...}+可以产生“上标引用的参考文献”,
如\supercite{Meta_CN,chen2007act,DPMG}。
使用\verb+\parencite{key1,key2,key3...}+则可以产生水平引用的参考文献,例如\parencite{JohnD,zhubajie,IEEE-1363}。
请看下面的例子,将会穿插使用水平的和上标的参考文献:关于书的\parencite{Meta_CN,JohnD,IEEE-1363},关于期刊的\supercite{chen2007act,chen2007ewi},
会议论文\parencite{DPMG,kocher99,cnproceed},
硕士学位论文\parencite{zhubajie,metamori2004},博士学位论文\supercite{shaheshang,FistSystem01,bai2008},标准文件\parencite{IEEE-1363},技术报告\supercite{NPB2},电子文献\parencite{xiaoyu2001, CHRISTINE1998},用户手册\parencite{RManual}。

总结一些注意事项:
\begin{itemize}[noitemsep,topsep=0pt,parsep=0pt,partopsep=0pt]
\item 参考文献只有在正文中被引用了,才会在最后的参考文献列表中出现;
\item 参考文献“数据库文件”bib是纯文本文件,请使用UTF-8编码,不要使用GBK编码;
\item 参考文献条目中通过language域是否为空判断是否是中文文献;
\end{itemize}

\section{用listings插入源代码}

原先ctexbook文档类和listings宏包配合使用时,代码在换页时会出现莫名其妙的错误,后来经高人指点,顺利解决了。
感兴趣的话,可以看看\href{http://bbs.ctex.org/viewthread.php?tid=53451}{这里}。
这里给使用listings宏包插入源代码的例子,这里是一段C代码。
另外,listings宏包真可谓博大精深,可以实现各种复杂、漂亮的效果,想要进一步学习的同学,可以参考
\href{http://mirror.ctan.org/macros/latex/contrib/listings/listings.pdf}{listings宏包手册}。

\begin{lstlisting}[language={C}, caption={一段C源代码}]
#include <stdio.h>
#include <unistd.h>
#include <sys/types.h>
#include <sys/wait.h>

int main() {
  pid_t pid;

  switch ((pid = fork())) {
  case -1:
    printf("fork failed\n");
    break;
  case 0:
    /* child calls exec */
    execl("/bin/ls", "ls", "-l", (char*)0);
    printf("execl failed\n");
    break;
  default:
    /* parent uses wait to suspend execution until child finishes */
    wait((int*)0);
    printf("is completed\n");
    break;
  }

  return 0;
}
\end{lstlisting}

\section{用algorithm和algorithmic宏包插入算法描述}

如算法\ref{algo:sum_I_want}所示。

\begin{algorithm}
\caption{求和算法}
\label{algo:sum_I_want}
\begin{algorithmic}
\REQUIRE $n \geq 1$                  %输入条件
\ENSURE $Sum = 1 + \cdots + n$       %输出
\STATE $Sum \leftarrow 0$            %\STATE 命名演示
\IF {$n < 1$}                        %条件语句
\PRINT {Input Error}                 %打印语句
\ENDIF
\RETURN $Sum$
\end{algorithmic}
\end{algorithm}

